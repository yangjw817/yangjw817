\documentclass{resume} % 设置字体大小为10pt
% \usepackage[margin=0.5in]{geometry} % 设置页边距为0.5英寸
\usepackage{zh_CN-Adobefonts_external} 
\usepackage{linespacing_fix}
\usepackage{cite}
\geometry{margin=0.85in} % 设置页边距为0.5英寸
\begin{document}
\pagenumbering{gobble}

\setstretch{1.29}

%***"%"后面的所有内容是注释而非代码,不会输出到最后的PDF中
%***使用本模板,只需要参照输出的PDF,在本文档的相应位置做简单替换即可
%***修改之后,输出更新后的PDF,只需要点击Overleaf中的“Recompile”按钮即可
%**********************************姓名********************************************
\name{杨嘉伟}
% {\fontsize{12pt}{12pt}\selectfont\name{杨嘉伟}}
%**********************************联系信息****************************************
%第一个括号里写手机号,第二个写邮箱
\contactInfo{(+86) 13044942229}{997882952@qq.com}
%**********************************其他信息****************************************
%在大括号内填写其他信息,最多填写4个,但是如果选择不填信息,
%那么大括号必须空着不写,而不能删除大括号。
%\otherInfo后面的四个大括号里的所有信息都会在一行输出
%如果想要写两行,那就用两次这个指令(\otherInfo{}{}{}{})即可
\otherInfo{性别:男}{民族:汉族}{政治面貌:中共党员}{}
% \otherInfo{来历:钱钟书《围城》}{}{}{}
%*********************************照片**********************************************
%照片需要放到images文件夹下,名字必须是you.jpg,如果不需要照片可以不添加此行命令
%0.15的意思是,照片的宽度是页面宽度的0.15倍,调整大小,避免遮挡文字
\yourphoto{0.12}
%**********************************正文**********************************************


%***大标题,下面有横线做分割
%***一般的标题有:教育背景,实习(项目)经历,工作经历,自我评价,求职意向,等等
\section{教育背景}


%***********一行子标题**************
%***第一个大括号里的内容向左对齐,第二个大括号里的内容向右对齐
% ***\textbf{}括号里的字是粗体,\textit{}括号里的字是斜体
\datedsubsection{\textbf{华中科技大学\ \ 硕士}\ \ 计算机科学与技术学院\ \ 计算机科学与技术\ \ \ \ \ \ \ \ GPA : \ \ 3.5/4}{2022年9月 - 2025年6月}
\datedsubsection{\textbf{武汉大学\ \ \ \ \ \ \ \ \ \ 本科}\ \ 生命科学学院\ \ \ \ \ \ \ \ \ \ \ \ \ \ \ \ \ \ 生物科学\ \ \ \ \ \ \ \ \ \ \ \ \ \ \ \ \ \ \ \ \ \ \ \ GPA : 3.54/4}{2018年9月 - 2022年6月}
\datedsubsection{\textbf{武汉大学\ \ \ \ \ \ \ \ \ \ 辅修}\ \ 计算机学院\ \ \ \ \ \ \ \ \ \ \ \ \ \ \ \ \ \ \ \ \ \ 计算机科学与技术\ \ \ \ \ \ \ \ GPA : 3.68/4}{2019年3月 - 2022年6月}

% \begin{tabular}{ l l r }
%     \textbf{华中科技大学\ 硕士} & 计算机科学与技术学院\ 计算机科学与技术 & GPA : 3.5/4 \\
%     \textbf{武汉大学\ 本科} & 生命科学学院\ 生物科学 & GPA : 3.54/4 \\
%     \textbf{武汉大学\ 辅修} & 计算机学院\ 计算机科学与技术 & GPA : 3.68/4 \\
% \end{tabular}

% \datedsubsection{计算机科学与技术学院,计算机科学与技术,\textit{硕士研究生}}{}

% %***********列举*********************
% %***可添加多个\item,得到多个列举项,类似的也可以用\textbf{}、\textit{}做强调
% \begin{itemize} [parsep=1ex]
%   \item \textbf{证书来源}:购买自爱尔兰商人
% \end{itemize}


% \datedsubsection{\textbf{武汉大学}}{2018年9月 - 2022年6月}
% \datedsubsection{生命科学学院,生物科学,\textit{本科}}{}
% \begin{itemize} [parsep=1ex]
%   \item \textbf{课程成绩}:高等数学(4.0/4.0)、概率论和数理统计(3.7/4.0)、机器学习与模式识别(4.0/4.0)
% \end{itemize}

% \datedsubsection{计算机学院,计算机科学与技术,\textit{辅修}}{}
% \begin{itemize} [parsep=1ex]
%   \item \textbf{课程成绩}:离散数学(4.0/4.0)、算法设计与分析(4.0/4.0)、数据库系统(4.0/4.0)
% \end{itemize}


\section{学术经历}

\datedsubsection{\textbf{全息存储多级长效编码与数据通道}(国家重点研发项目 No. 2018YFA0701800)}{2022年9月 - 2024年9月} %No. 2018YFA0701800 -\ 科研骨干
\begin{itemize} [parsep=0.5ex]
% QC-LDPC编码器相关工作。使用C/C++以及优化高层次综合(HLS),在FPGA上实现近Tbps级别吞吐率的QC-LDPC编码器,通过对编码器的结构设计实现了3级并行和2级流水的编码计算操作,实现5.89×的单位硬件效率提升和154.5×的最高编码吞吐率提升,实现了922.66Gbps的编码效率。发表论文一篇(ICCD24' \textbf{CCF-B会议},已被接收,第一作者)
  \item \textbf{QC-LDPC编码器},使用C/C++以及优化高层次综合(HLS),在FPGA上实现近Tbps级别吞吐率的QC-LDPC编码器,通过对编码器的结构设计实现了3级并行和2级流水的编码计算操作,实现5.89×的单位硬件效率提升和154.5×的最高编码吞吐率提升,实现了922.66Gbps的编码效率。发表论文 HEncode: A Highly Modularized and Efficient FPGA QC-LDPC Encoder Using High Level Synthesis(ICCD24' \textbf{CCF-B会议},已被接收,第一作者)
  \item \textbf{QC-LDPC译码器},优化HLS在任务级调度中的表现,使用双缓冲区和重构输出缓冲优化译码器中译码单元运行时间不对齐的问题,综合吞吐率较SOTA提升5×
  参与论文:eLDPC: An Elastic HLS-friendly LDPC-Decoder with Dynamical Scheduling
  \item \textbf{数据通道与布局},使用了多层编码交织的数据布局提高了同轴全息光存储中原始数据的可靠性、存储效率和读写性能,构造吞吐率提升75\%,失效重构时间降低76\%
\end{itemize}

\datedsubsection{\textbf{GNN优化加速}(光明实验室:Transformer 模型异构硬件推理加速) -\ 研究人员}{2025年1月 - 至今} % 2026年12月
\begin{itemize} [parsep=0.5ex]
  \item 设计并实现了GNN训练框架HeteroGNN。深入分析了GNN训练过程中潜在的并行性和任务分配机会,提出了数据依赖感知的阶段划分策略,将训练过程从粗粒度的两个阶段细化为六个阶段,实现了阶段间的并行 。
  \item 设计了细粒度的计算划分模式和自适应任务调度器,根据拓扑结构的特性,在CPU和GPU上合理地分配计算任务,有效提升了GPU的利用率 。实验结果表明,HeteroGNN相比于DGL和PyG,实现了1.3倍到2.06倍的端到端训练加速。发表论文 HeteroGNN: A Heterogeneous Stage Division Based GNN Training Framework to Maximize CPU-GPU Parallelism(ICME25' \textbf{CCF-B会议},已接收,第三作者)
\end{itemize}

\datedsubsection{\textbf{筛选黑色素瘤致癌基因}(武汉大学病毒学国家重点实验室)\ -\ 数据分析员}{2022年3月 - 2022年5月}
\begin{itemize} [parsep=0.5ex]
\item 利用支持向量机对测序数据初步验证,使用随机森林算法,LASSO回归和递归特征消除算法对进行特征提取并最终确定8个潜在的黑色素瘤的相关基因,其中部分已被证实存在相关性,也存在尚未背研究的基因,提示后续的实验开展方向
\end{itemize}
% \datedsubsection{\textbf{GNN优化加速}}{}
% \begin{itemize} [parsep=0.5ex]
%   \item 优化GNN训练框架,分析识别GNN模型训练过程中潜在的并行性和任务分配机会,指导GPU-CPU异构硬件架构下训练框架的设计,将训练过程中常见的两阶段细粒度划分为六阶段,并利用拓扑结构的特性实现自适应工作负载平衡器,最终较常用的DGL和PYG性能提升约1.3-2.06倍。HeteroGNN: An Task Division Schema and Workload Balancer for GNN Training on CPU-GPU Architecture(第二作者)
% \end{itemize}


\section{实习经历}

\datedsubsection{\textbf{深圳市腾讯计算机系统有限公司\ -\ 技术工程事业群(TEG)}\ \ 后台开发}{2024年5月 - 2024年9月}
% 一、边缘机房采集系统和管理系统开发,基于腾讯内部trpc-go框架搭建系统,用于收集并统计20k+台服务器的10s级流量采样监控数据,通过IO聚合持久化至pgsql中,通过timescale插件管理4TB的原始采样时序数据,根据多维度将原始数据汇总为5min级的结果数据并计算对应的95成本线,用于内部数据平台展示;二、网络质量探测功能开发:1. 机房内部连通性探测功能,用于检测机房内部服务器连接丢失情况;2. OC与三通机房之间网络质量的探测,网络质量数据用于95计费调度;三、内部接口需求开发,基于trpc-go框架实现2个接口,提供内部多个模块中超过50k ip相关信息的1k QPS的查询服务,多服务器容灾部署,并接入现有系统。
% \datedsubsection{后台开发}{广东省深圳市}
\begin{itemize}[parsep=0.5ex]
  \item 边缘机房采集系统和管理系统开发,基于腾讯内部trpc-go框架搭建了该系统,用于收集并统计20k+台服务器的10s级流量采样监控数据,通过IO聚合持久化至pgsql中,通过timescale插件管理4TB的原始采样时序数据,根据机房和端口维度将原始数据汇总为5min级的结果数据并计算对应的95成本线,在内部数据平台展示,总结整理CDN相关95计费调度近5年相关工作并整理为技术文档。
  \item 网络质量探测功能开发,在采集-管理系统的基础上实现1. 机房内部连通性探测功能,用于检测机房内部服务器连接丢失情况;2. OC与三通机房之间网络质量的探测,网络质量数据用于95计费调度
  \item 内部接口需求开发,基于trpc-go框架实现2个接口,提供内部多个模块中超过50k ip相关信息的1k QPS的查询服务,多服务器容灾部署,并接入现有集成系统。
\end{itemize}

\datedsubsection{\textbf{四川省成都市都江堰市蒲阳街道办}\ 政务见习}{2023年6月 - 2023年7月}
% 一、协助处理了暑期两个月汛期的值班排班、整理委托流转农用土地信息等工作,也了解了河长制相关的一些情况,感受到了党和政府对于人民的人身安全和财产安全的关切,体现了为人民服务的宗旨;二、通过实地参观茶溪谷产业园,我更深入地了解和认识到这里以“三元共生”机制实施“农文旅融合”发展战略的成功经验。
\begin{itemize}[parsep=0.5ex]
  \item 协助处理汛期值班排班、农用土地流转信息等工作,并了解河长制相关情况,体会到政府对保障人民生命财产安全的重视,以及为人民服务的宗旨;。
  \item 参观茶溪谷产业园,深入了解其以“三元共生”机制实施“农文旅融合”发展战略,学习以本地文化和自然资源推动乡村振兴的经验。
  \item 践行“我为群众办实事”精神,开展了安全主题宣讲等系列活动,引导未成年人增强安全意识、提升自我保护能力。
\end{itemize}

% \datedsubsection{\textbf{武汉大学病毒学国家重点实验室 - 数据分析}}{2022年2月 - 2022年5月}
% % \datedsubsection{业余科研}{湖北省武汉市}
% \begin{itemize}[parsep=0.5ex]
%   \item 利用支持向量机对测序数据初步验证,使用随机森林算法,LASSO回归和递归特征消除算法对进行特征提取并最终确定8个潜在的黑色素瘤的相关基因,其中部分已被证实存在相关性,也存在尚未背研究的基因,提示后续的实验开展方向
% \end{itemize}

\section{学生干部经历}
\datedsubsection{\textbf{华中科技大学计算机科学与技术学院\ 存储所研究生第一党支部}\ 党支部书记}{2023年11月 - 2024年11月}
\begin{itemize}[parsep=0.5ex]
\item 累计转正\textbf{6名}中共正式党员,负责培养\textbf{9人次}入党积极分子,负责了\textbf{一次党支部拆分}工作.
\item 抓紧抓牢三会一课和主题党日的开展。积极推进支部坚强战斗堡垒作用,主动邀请院校讲学团走进支部讲党课,鼓励党员走出去学习党史校史,参加各类志愿活动。 % 担任书记期间,
\end{itemize}

\datedsubsection{\textbf{华中科技大学计算机科学与技术学院\ 研究生会实践部}\ 学生干事-实践部部长}{2022年9月 - 2024年9月}
\begin{itemize}[parsep=0.5ex]
\item 开展“红旗下的IT人”系列志愿行活动,线上志愿服务群面向校内师生\textbf{近2000人},线下常态化开展线下开放日活动,对接校内师生的电脑软硬件维修服务,同时走出校园开展一院一社等社区志愿服务,活动曾受到人民日报和校级微信公众号报道。
\item 参与过校园\textbf{十大提案}并成功入选,帮助同学们缓解了电动车充电难的问题
\end{itemize}

\datedsubsection{\textbf{华中科技大学计算机科学与技术学院\ 硕2201班}\ 团支部书记}{2022年9月 - 2023年9月}
\begin{itemize}[parsep=0.5ex]
\item 引导同学们主动向党组织靠拢,在第一学年就有\textbf{7位团员}提交了入党申请书,占支部团员总人数的\textbf{30\%}。
\item 负责学生政治思想的学习工作,积极开展各项主题团日活动,团结同学,使同学们端正政治方向,重视每周青年大学习,使得本班青年大学习完成率超学院平均水平。
\end{itemize}

\datedsubsection{\textbf{武汉大学生命科学学院\ 2018级1班}\ 班长}{2018年9月 - 2022年6月}
\begin{itemize}[parsep=0.5ex]
\item 在\textbf{4年}任职期间恪尽职守,在辅导员的指导下,组织协调班委和同学,开展各项班级活动和管理工作。主动开展各类特色活动,如入学破冰活动,疫情居家的厨艺比拼活动等,\textbf{三次}位居学院班级活动评比前三。
\item 与支书共同积极推动班团一体化建设,因此也曾获评先进团支部。
\end{itemize}

\section{证书\&荣誉}
% \begin{itemize}[parsep=0.5ex]
%     \item 掌握工具:C/C++,Go,数据库,软硬件结合加速等
%     \item 学生工作:硕-存储所研究生第一党支部书记;硕-院研会实践部部长;本-生命科学学院1801班班长
%     \item 证书/荣誉:CET-6 CET-4;第十三届全国大学生数学竞赛(非数学类)三等奖;华中科技大学三好研究生,华中科技大学优秀团员,武汉大学优秀毕业生,武汉大学优秀学生干部等
% \end{itemize}

\begin{itemize}[parsep=0.5ex]
    \item \textbf{证书}:CET-6,CET-4,第十三届全国大学生数学竞赛(非数学类)三等奖 CMS(鄂)F20212999;
    \item \textbf{发明专利1}\ :基于高层次综合的QC-LDPC编码器、通讯设备及存储产品(CN202410759357 授予);
    \item \textbf{发明专利2}\ :一种自动上下台阶的机构、控制方法、控制器及系统(CN202310823382 申请);
    \item \textbf{软件著作权}\ :基于FPGA的QC-LDPC编码器软件 V3.0(证书号:软著登字第13714687号);
    \item \textbf{实用新型专利}\ :一种简易的防吸入沉淀的移液枪枪头(ZL 2020 2 2090314.9 授予);
    %发明专利两项[CN202410759357(授予),CN20231082338(审查)],实用新型专利一项[CN2020220903149(授予)]
    \item \textbf{荣誉}\ :\textbf{国家奖学金}(2024.12),华中科技大学三好研究生(2024.12, 2023.12),华中科技大学优秀共青团员(2023.5),武汉大学优秀毕业生(2022.5),武汉大学优秀共青团干(2021.5, 2020.5),武汉大学优秀学生(2021.12, 2020.12, 2019.12),武汉大学优秀学生干部(2019.6)

    % 华中科技大学三好研究生(2023),华中科技大学优秀共青团员(2023),武汉大学优秀毕业生(2022),武汉大学优秀共青团干(2021, 2019),武汉大学优秀学生(2021, 2020, 2019),武汉大学优秀学生干部(2020)等
    \item \textbf{奖学金}\ :华中科技大学科技创新奖学金(2024.12),华中科技大学知行优秀奖学金三等(2023.12),华中科技大学学业奖学金二等(2024.12, 2023.12 2022.12),武汉大学学业奖学金乙等(2021.12),武汉大学学业奖学金丙等(2020.12 2019.12)
\end{itemize}
% \section{学生工作}
% \datedsubsection{华中科技大学计算机学院\ 存储所研究生第一党支部\ 支部书记}{2023年11月 - 今}
% \datedsubsection{华中科技大学计算机学院\ 研究生会\ 实践部部长}{2023年9月 - 今}
% \datedsubsection{武汉大学\ 2018级1班\ 班长}{2018年9月 - 2022年6月}


% \section{证书/荣誉}

% \begin{itemize}[parsep=0.5ex]
%   \item CET-6 CET-4
%   \item 第十三届全国大学生数学竞赛(非数学类)三等奖
%   % \item 华中科技大学三好研究生,武汉大学优秀毕业生,武汉大学优秀共青团干等
%   \item 华中科技大学三好研究生(2023),华中科技大学优秀团员(2023),武汉大学优秀毕业生(2022),武汉大学优秀共青团干(2019,2021),武汉大学优秀学生干部(2020),武汉大学优秀学生(2019,2020,2021)
% \end{itemize}

\end{document}
